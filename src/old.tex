% $Id: unfug.tex 2 2005-09-12 16:22:45Z pfeifer $
% Copyright 2005
% Distributed under the terms of the GNU General Public License v2
\documentclass[10pt,%
               landscape,%
               a4paper
]{article}

\usepackage[T1]{fontenc}
\usepackage[latin1]{inputenc}
\usepackage{ngerman}
\usepackage{graphicx}

\pagestyle{empty}

\begin{document}

% METAFONT Zeichensatz!
\font\logo=lcmssb8 scaled 23000
\font\themafont=lcmssb8 scaled 5000
\font\deffont=lcmssb8 scaled 2500
\font\authorfont=lcmssb8 scaled 3500
\font\astafont=lcmssb8 scaled 1000

\setlength{\unitlength}{1mm}

\begin{picture}(0,0)


	%% next paragraph: unfug logo and header
  \put(-50,25){\deffont Unix Friends and User Group}
  \put(-50,13){\deffont http://www.unfug.org -- news://fhf.org.unfug}
	\put(-85,-40){{\logo UnFUG}}
	\put(160,25){\includegraphics[scale=2.5]{images/logo}}
  \put(160,20){\astafont Wintersemester 06/07}

	%% Datum und Oertlichkeit von UnFUG
  \put(-50,-60){\deffont Donnerstag 18-01-2007, 19:00 Uhr -- B1.32}

	%% Titel
  \put(-50,-80){\themafont Pimp My *nix}

%  \put(-50,-100){\authorfont verschiedene Referenten}
	%% subtitles
	%% down here we can move entries left, right, up and down
	%% all upper entries _are_ the unfug logo (cooperate design! ;-)
  \put(-50,-105){\deffont -- MMX, SSE, Caches und Optimierungen;}
  \put(-50,-115){\deffont -- ArchLinux; nuetzliche Debian-Tools;}
  \put(-50,-125){\deffont -- rlwrap; git; Unison;}
  \put(-50,-135){\deffont -- LUKS; \ldots}
%  \put(-50,-150){\deffont -- }

\end{picture}
\end{document}

HELPDESK:

 - zum Unterstreichen von Textpassagen ist folgendes Makro zu
   bevorzugen: \underline (z.B. \underline{23-03-2005, 19:00 Uhr})

 - lcmssb8 ist strikt 7bit clean -> keine Implementierung von Krautlauten
   man metafont ;-)

% vim:set ts=2 noet tw=120 ft=tex:
